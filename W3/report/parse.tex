\section{Parsing}

\subsection{Partwise parsers}
To allow for nore readable code the parser is split into different smaller
parsers/methods. Topmost in the file we have a number of smaller convenience
parsers such as \texttt{charToken}, \texttt{stringToken}, \texttt{number} and so
forth. These are meant to catch and parse small components that are likely to be
used by severeal other parsers.

The main parser is the one that parses ``programs'', it is called form
\texttt{parseString} and will in turn call the rest of the parsers (indirectly
of course, since it only really calls the \texttt{defs} parser itself. The code
for the method can be seen in Figure \ref{code:progparser}.

\codefig{progparser}{../CurvySyntax2.hs}{89}{94}{The implementation of the
  \texttt{prog} method which parses programs/lists of definitions.}

This pattern is repeating through the whole parser combinator ``tree'' where a
general parser will call smaller parsers on the sub components, for instance
\texttt{prog} will call the \texttt{defs} parsers which calls the \texttt{def}
parser and so on. The code can be seen in Figure \ref{code:defdefs}.


\codefig{defdefs}{../CurvySyntax2.hs}{96}{106}{Parse a list of definitions.}

Parsers suffixed with ``op'' are parsers which handle ``operators'' for a
different parser, for instance \texttt{expr} have \texttt{exprop} which will
handle the operators in expressions. This can be seen in Figure
\ref{code:exprexprop}

\codefig{exprexprop}{../CurvySyntax2.hs}{183}{211}{Parse expressions.}

There is more code, but this shold give a look into how our code is constructed.


\subsection{\texttt{parseString}}
Uses readP to parse a string with the parsers defined earlier in the program.

\codefig{parsestring}{../CurvySyntax2.hs}{226}{231}{The implementation of the
  \texttt{parseString} method.}



\subsection{\texttt{parseFile}}
\texttt{parseFile} was implemented with the suggestion from the assignment text
and can be seen in Figure \ref{code:parsefile}.

\codefig{parsefile}{../CurvySyntax2.hs}{233}{235}{The implementation of the
  \texttt{parseFile} method.}
