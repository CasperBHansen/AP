\section{Testing}

As before we did a number of just random testing, and then constructed a number
of larger tests for the report. The tests and code for running them can be seen
in Figure \ref{code:tests}.


Testing is done simply by loading the module and running the \texttt{runTests}
method. This will run 6 different tests and print the result like so:
\begin{verbatim}
0=OK
1=OK
2=OK
3=OK
4=OK
5=OK
\end{verbatim}

If one of the tests fail it will read ``FAIL'' instead of ``OK'' in the list.
The tests are designed so they test a number of things such as the precedence of
operators, assignments and expressions.

\codefig{tests}{../CurvySyntax2.hs}{238}{259}{Method for running all our tests.}
