\section{Manipulative Functions}
The following subsections each pertain to a seperate function created to perform
the actions of an instruction decoded from a MSM program.

\subsection{\texttt{push}}
\codefig{funpush}{../MSM.hs}{116}{119}{Function implementation}

This function takes an integer as input and puts it on top of the stack, this is
done by accessing the state of the MSM monad and appending it to the front of
the stack list.

\subsection{\texttt{pop}}
\codefig{funpop}{../MSM.hs}{121}{129}{Function implementation}

\texttt{pop} will access the state of the monad and retrieve the stack, if the
stack is an empty list it will fail with the \texttt{StackUnderflow} error, if
it is not, it will remove the first element and return it. The initial type
decleration did not contain this, but it helps making some later functions
(\texttt{store}, \texttt{load}, \texttt{neg}, \texttt{add}, \texttt{jmp} and
\texttt{cjmp}) easier to write since they can use \texttt{pop} to retrieve data
from the stack.

\subsection{\texttt{dup}}
\codefig{fundup}{../MSM.hs}{131}{134}{Function implementation}

\texttt{dup} uses simple pattern matching to take the first element of a list
and copy it.

\subsection{\texttt{swap}}
\codefig{funswap}{../MSM.hs}{136}{139}{Function implementation} \texttt{swap}
uses simple pattern matching to re arrange the forward element of the stack.

\subsection{\texttt{newreg}}
\codefig{funnewreg}{../MSM.hs}{141}{144}{Function implementation}

To insert a new entry into the \texttt{Map} we use \texttt{Map.insert} to insert
it to the appropriate place, the data is initialized as a zero.

\subsection{\texttt{store}}
\codefig{funstore}{../MSM.hs}{146}{153}{Function implementation} \texttt{store}
pops the two top elements of the stack, and checks if the corresponding register
key, if there is no register with the key it will fail with the
\texttt{UnallocatedRegister} error, else it will insert the key in the map and
replae it in the monad.

\subsection{\texttt{load}}
\codefig{funload}{../MSM.hs}{155}{161}{Function implementation}

\texttt{load} pops a key from the stack and tries to look it up in the register
\texttt{Map}, if the method returns \texttt{Nothing} it means the register was
unallocated and it fails with the \texttt{UnallocatedRegister} error. Otherwise
it pushes the returned value unto the stack using the \texttt{push method.}

\subsection{\texttt{neg}}
\codefig{funneg}{../MSM.hs}{163}{166}{Function implementation}

This method simply pops a stack entry and negates it before pushing it back on
the stack.

\subsection{\texttt{add}}
\codefig{funadd}{../MSM.hs}{168}{172}{Function implementation}

The \texttt{add} method pops two numbers from the stack and adds the together
before pushing the result back on the stack.

\subsection{\texttt{jmp}}
\codefig{funjmp}{../MSM.hs}{174}{178}{Function implementation}

\texttt{jmp} pops an integer from the stack and sets it as the value of the
\texttt{PC} in the state of the monad.

\subsection{\texttt{cjmp}}
\codefig{funcjmp}{../MSM.hs}{180}{188}{Function implementation}

This method pops a value from the stack, checks if it's less than zero, if it is
it will push the argument $i$ unto the stack and call \texttt{jmp}. If the value
is 0 or above, it will simply advance the program counter.

\subsection{\texttt{write}}
\codefig{funcjmp}{../MSM.hs}{111}{114}{Function implementation}

The is the beginning of the optional \texttt{write} assignment. It was not
finished.  Currently it will simply pop a value from the stack and return it as
a string.
