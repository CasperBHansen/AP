\section{MusicXMatch}

The second part of the assignment was usage of our mapreduce framework to
collect statistics and search through the Million Sond Dataset.  For our tests
we used the small dataset (\textit{data/mxm\_dataset\_test.txt}) in order to
speed up test times, and spare our puny laptops.

\subsection{Task 1}
The first task was to compute the total number of words in all the songs
together. For this task we do a little preprocessing, we take seperate the
wordlist and the list of tracks. See Figure \ref{code:preproc}.

\codefig{preproc}{../assignment/tests.erl}{27}{31}{Parsing the dataset file into
  a list of words, and a list fo tracks.}

The mapper function will then use the track parsing function from
\texttt{read\_mxm} to get the list of words used in the track and sums their
occurence counters. Thus sum is then sent to the reducer.

The reducer function is simply the sum function from the testing examples
given. The calls for the test can be seen in Figure \ref{code:mxm1}.

\codefig{mxm1}{../assignment/tests.erl}{31}{40}{Using our map reduce
  framework to count the total number of words in all the songs..}

\subsection{Task 2}
Our code for calculating the avergage number of different words per song, and
the average numer of words per song can be seen in Figure \ref{code:mxm2}.

\codefig{mxm2}{../assignment/tests.erl}{44}{61}{MR job for solving the task.}

The mapping function will return the number of different words used in a single
track along with the number of words used totally. It also returns a $0$ in
order to allow for a counter to be implemented in the reducer.

The reducer calculates the running average of both numbers returned by the
mappers and increase the counter as it goes along.

\subsection{Task 3}
For testing our \texttt{grep} function we created two cases, one that should
return nothing at all and one that returns a few ID's. They can be seen in
Figure \ref{fig:mxm3}.

\codefig{mxm3}{../assignment/tests.erl}{65}{72}{Our two tests of \texttt{grep}.}

\texttt{grep} is implemented as Seen in Figure \ref{code:grep}.

\codefig{grep}{../assignment/part2.erl}{19}{40}{Our \texttt{grep}
  implementation.}


\codefig{inwords}{../assignment/part2.erl}{12}{15}{Helper function for
  \texttt{grep}.}

The mapper function checks if the specified word is in the list of words used in
a single track and returns the ID's of the songs, otherwise it returns an empty
list.

The reducer simply appends the lists from the mappers and returns the result.

\subsection{Task 4}
We tried to implement a \texttt{dict} in \textit{../assignment/revindex.erl} seen in Figure \ref{code:dict}.

\codefig{dict}{../assignment/tests.erl}{74}{88}{Code that constructs a dictionary using our map reduce framework.}

As can be seen on the figure the code is out-commented. Since the running time was several hours for constructing the entire index.

\subsection{Task 5}
While the preliminary initialization of the reverse index dictionary might
exceed the algorithmic complexity of the \code{grep} method, look-ups using
the reverse indexing method allows us to retreive data in constant time once
this has been built. Conversely, \code{grep} has a linear time complexity at
all times.
