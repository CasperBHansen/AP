\documentclass[a4paper,11pt]{article}
\usepackage{a4wide}
%\usepackage{appendix}
\usepackage{blkarray}
\usepackage[utf8x]{inputenc}
\usepackage{ucs}
\usepackage[T1]{fontenc}
\linespread{1.2}
\usepackage{amsmath,amssymb,amsthm,amsfonts,ulem}
\usepackage{courier}
\usepackage{color}
\usepackage{mdwlist}
\usepackage{multicol}
\usepackage{verbatim}
%\setcounter{secnumdepth}{1}
%\setcounter{tocdepth}{3}
\usepackage{todonotes}
\usepackage{hyperref}
\usepackage{listings}

\usepackage{subcaption}
\usepackage{float}
\usepackage{mdwlist}
\usepackage{wrapfig}
\usepackage{caption}
\newcommand{\graphicc}[4]{\begin{figure}[H] \centering
            \includegraphics[width={#1\textwidth}, keepaspectratio=true]{{#2}}
            \caption{{#3}} \label{#4} \end{figure}}

\newcommand{\figref}[1]{(see figure~\ref{#1})}

\newcommand{\code}[1]{{\tt #1}}
\newcommand{\file}[1]{{\tt #1}}

\newcommand{\assignment}{../assignment/mr.erl}
\newcommand{\tests}{../assignment/tests.erl}

% usage: \codefig{label}{file}{firstline}{lastline}{description}
\newcommand{\codefig}[5]
{
\begin{figure}[H]
    \lstinputlisting[firstnumber=#3,firstline=#3,lastline=#4]{#2}
    \caption{#5 (#2)}
    \label{code:#1}
\end{figure}
}

\definecolor{comment}{rgb}      {0.38, 0.62, 0.38}
\definecolor{keyword}{rgb}      {0.10, 0.10, 0.81}
\definecolor{identifier}{rgb}   {0.00, 0.00, 0.00}
\definecolor{string}{rgb}       {0.50, 0.50, 0.50}

\lstset
{
    language=erlang,
    % general settings
    numbers=left,
    frame=single,
    basicstyle=\footnotesize\ttfamily,
    tabsize=2,
    breaklines=true,
    showstringspaces=false,
    % syntax highlighting
    commentstyle=\color{comment},
    keywordstyle=\color{keyword},
    identifierstyle=\color{identifier},
    stringstyle=\color{string},
}

\renewcommand{\contentsname}{Tasks}

\title
{
    {\Large Weekly Assignment 5} \\
    Advanced Programming 2014 @ DIKU
}

\author
{
    Martin Jørgensen \\
    University of Copenhagen \\
    Department of Computer Science \\
    {\tt tzk173@alumni.ku.dk}
    \and
    Casper B. Hansen \\
    University of Copenhagen \\
    Department of Computer Science \\
    {\tt fvx507@alumni.ku.dk}
}

\date{\today}

\begin{document}

\maketitle
\thispagestyle{empty}
\begin{multicols}{2}
    \begin{abstract}
        For this assignment we implement a so-called map-reduce framework,
        which is a technique used to distribute the workload over a number of
        running threads. Each thread processes a bit of the input in the
        mapper-threads whilst the reducer-thread waits and collects the
        processed output. Upon receiving all of the mapper output data, the
        reducer then reduces this into a single output.
    \end{abstract}
    \vfill{\ }\columnbreak
    \tableofcontents
\end{multicols}
\clearpage

\section{Map-Reduce Framework}
\subsection{Coordinator}
\label{coordinator}
The coordinator is at the heart of the map-reduce framework, as it takes job
requests (line 71) and prepares the reducer and mappers, and is responsible
for handing out the data and starting the processing. First, we give the
reducer the necessary initialization information, which makes it wait for data
segments from the mappers. On line 74 it sends out the function used by the
mappers to process each data segment, using helper the function
\code{send\_func} (see appendix \ref{helper:send_func}) we've provided. It
then sends out the associated job data (line 75), using helper the function
\code{send\_data} provided by the assignment code skeleton. This effectively
initializes the running mapper threads.

Since the coordinator is the thread responsible for reporting back to the job
request with an appropriate response, we decided to prepare the framework for
some simple error-handling by the \code{receive}-statement on lines 76--79,
where we wait for the reducer response, effectively a blocking call as we must
wait for the reducer to receive and process all of the data --- although we do
not have any thrown errors anywhere.

\codefig{coordinator}{\assignment}{63}{85}{Coordinator loop}

When the reducer replies, it should match the tuple \code{ok} and the result.
This case makes the coordinator respond back to the requester, also with a
tuple \code{ok} followed by the result.

Lastly, line 80 ensures that upon having processed a job, we wait for further
instructions.

\subsection{Mapper}
As described, initialization of the mapper threads happens as a consequence of
the coordinator sending out a mapper function (see section \ref{coordinator}).
Upon receiving this function (lines 150--151), we simply restart the loop
substituting the old function with this newly supplied function.

\codefig{mapper}{\assignment}{140}{156}{Mapper loop}

Upon receiving data (line 146), we send the processed data segment to the
reducer on line 147, and then wait for further instructions on line 148.

\subsection{Reducer}
The reducer is responsible for gathering incoming processed data segments from
the mapper threads, and turn all of these into one single result. We do so by
handing the reducer a message (line 115) letting it know it should start
collecting this data, and within it all of the necessary informations to
produce such an output; the reduction function, an initial value and the
number of expected incoming data segments.

Upon receiving this message, we pass control of the reducer thread to the
\code{gather\_data\_from\_mappers} function (see figure \ref{code:gatherer}).
Once all data has been retreived this function call returns allowing the
reducer to wait for further instructions on line 118.

\codefig{reducer}{\assignment}{109}{125}{Reducer loop}

...

\codefig{gatherer}{\assignment}{127}{136}{Gatherer}


\appendix
%
% appendix.tex
%

\section{Results}

\subsection{Broadcast test}
\label{test:broadcast}
\codefig{test:broadcast-1}{broadcast-test.txt}{1}{37}{Output of broadcast test}
\codefig{test:broadcast-2}{broadcast-test.txt}{38}{82}{Output of broadcast test}


\end{document}
