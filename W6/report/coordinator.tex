\subsection{Coordinator}
\label{coordinator}
The coordinator is at the heart of the map-reduce framework, as it takes job
requests (line 71) and prepares the reducer and mappers, and is responsible
for handing out the data and starting the processing. First, we give the
reducer the necessary initialization information, which makes it wait for data
segments from the mappers. On line 74 it sends out the function used by the
mappers to process each data segment, using helper the function
\code{send\_func} (see appendix \ref{helper:send_func}) we've provided. It
then sends out the associated job data (line 75), using helper the function
\code{send\_data} provided by the assignment code skeleton. This effectively
initializes the running mapper threads.

Since the coordinator is the thread responsible for reporting back to the job
request with an appropriate response, we decided to prepare the framework for
some simple error-handling by the \code{receive}-statement on lines 76--79,
where we wait for the reducer response, effectively a blocking call as we must
wait for the reducer to receive and process all of the data --- although we do
not have any thrown errors anywhere.

\codefig{coordinator}{\assignment}{63}{85}{Coordinator loop}

When the reducer replies, it should match the tuple \code{ok} and the result.
This case makes the coordinator respond back to the requester, also with a
tuple \code{ok} followed by the result.

Lastly, line 80 ensures that upon having processed a job, we wait for further
instructions.
