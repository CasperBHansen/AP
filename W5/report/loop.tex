\section{Main loop}
This section covers the main loop. Since this function is big and clearly
segmented, we will cover it case for case. For the full implementation either
consult the \textit{facein.erl} file or see Figure \ref{code:fullloop}.

\texttt{loop} takes a triple as argument, the triple contains the name of the
person, the list of their friends and a list of messages the person have
recieved. The loop will wait to recieve a message and then depending on pattern
matching will performs actions as descriped in the following subsections.

\subsection{Adding friends}
Adding a friends is a 2 step process, as descriped in \texttt{add\_friend} the
person we want to add ($F$) to a friendslist ($P$s friendlist), recieves a
message with a pattern as shown in \ref{code:add}.

\codefig{add}{\assignment}{39}{46}{The pattern that catches the first step of a
  friend request.}

When the process recieves the proper message it will send a message to $P$ with
it's own PID and name and then await a reply from $P$. The reply will the be
forwarded back to the caller. In the end it will call itself (\texttt{loop})
with it's own name, friends and messages.

\codefig{name}{\assignment}{48}{54}{Adding a friend to a friendlist and
  responding.}

The second step is in $P$ which matches a message with the pattern shown in
Figure \ref{code:name} it will check if $F$ is already on $P$'s friendlist, if
it is it will send an error back, otherwise it will send an \texttt{ok} back and
then call \texttt{loop} with it's name, it's friendlist with $F$ appended and
the message list.

\subsection{Retrieving friends}

\codefig{friends}{\assignment}{56}{59}{Retrieving the friendlist and sending it
  back.}

When a message matches the pattern seen in Figure \ref{code:friends} it will
respond with a message containing it's ID a friend list before it restarts the
\texttt{loop} method with the same arguments.

\subsection{Broadcasting a message}
\subsection{Retrieving messages}
\subsection{Invalid message}