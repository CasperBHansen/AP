%
% 2/solution.tex
%

\subsection{Preliminary}
In order to start implementation of anything in the interpreter, I had to
define the basic types (see appendix \ref{code:interp|part:types0} and
\ref{code:interp|part:types1}) used manipulated throughout the implementation.
I reason that a simple \code{Map} synonym would do for the \code{Store} as
they are basically synonomous in nature.

I reason that a global state should contain a store of references to objects
(line 52). A store of methods (line 53) bound by class name and its own name
for lookup at runtime, since these aren't bound by instantiations, and can be
reused across many instances. An output (line 54), which is merely a string
and a unique ID counter (line 55), for allocating of unique IDs.

\subsection{Solution}
With the types defined I then had the means to define the \code{FastM a}
monad, which I defined as given a program \code{Prog} and a state
\code{GlobalState} produces either an error or a state and a computed value
\code{(GlobalState, a)}. The binding operator is defined as given the inputs
of the monad type (a program and a state) compute the result, and apply the
monadic function \code{f}, returning the transformed monad. The return
operation merely wraps the given argument, and the fail operation is as simply
defined as it can be, using \code{error}. This monad is responsible for
handling the runtime of the program being executed.

Having the \code{FastM} monad, I proceeded to define the monadic manipulation
functions (see appendix \ref{code:interp|part:fastm-manip}), which I believe
are all very self-explanatory from their naming.

Defining the \code{FastMethodM a} as taking an object reference
\code{ObjectReference}, and a method state \code{MethodState}, manipulates the
runtime (in other words, producing a \code{FastM a}). The binding operator is
defined as given the inputs of the monad type (a reference and a method state)
compute the result, and apply the monadic function \code{f}, returning the
transformed result. The return and fail operations are both feed to the
lifting function (lines 199--200), which lifts the \code{FastMethodM} to
perform the operation as if it were a \code{FastM}.

As the case with \code{FastM}, the definitions and workings of the monadic
manipulation functions (see appendix \ref{code:interp|part:fastmethodm-manip})
of \code{FastMethodM} are all very self-explanatory from their naming.

An interesting monadic function is the \code{evalMethodBody} which given an
object reference, a list of arguments and a method body (a list of
\code{Expr}) alters the runtime state. Firstly, it binds the given arguments,
and it then proceeds to execute the method body. Also, the \code{evalArgs}
is quite handy, as it allows for easy argument processing. It simply maps over
the arguments using the monadic function \code{evalExpr}, producing a list of
corresponding values.

The last, I believe, correctly implemented function is \code{findClassDecl},
which given a name looks up the corresponding class declaration. If it does
not yield any results or if more than one such name exists, a run-time error
is produced. Otherwise we can register the contained methods of the class and
return the class definition for instantiation (see subsection Reflections on
this matter).
