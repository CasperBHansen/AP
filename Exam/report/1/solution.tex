%
% 1/solution.tex
%

% omissions
% design decisions

\subsection{Solution}
In designing the parser, I drew upon the fact that the Haskell parser library
used (ReadP) allows for building smaller parsers. I began the process by
implementing a subset of the language; namely the expression grammar
(excluding {\it Expr '.' Name '(' Args ')'}, which is an omission I have made
because of lack of time). Each form an expression could take on was defined as
a parser in its own right.

Once I was able to parse a selection of expressions, I decided to move on, as
the expression parser was stable and adding new expression types was a breeze.
I then moved on to implementing a parser for the {\it Params} and {\it Exprs}
grammar production rules. These were fairly easy, as a result of having the
convenience functions handy --- all I had to do was to get the contents of the
parens, which the \code{sepBy} parser was an obvious choice for.

With these done, I had the means to implement the remainder of the expression
parsers, as some of them depended upon the formerly described functionality,
and this concludes the expression parser.

Being able to parse expressions I could now focus on implementing the class
declaration parser. Because I couldn't immediately see how to modularize the
methods I decided to simply write a parser for each kind, so as to not spend
too much time on figuring it out. Some erroneous parts occurs in this parser
(see reflections and assessment).

Having all of the sub-parsers in place, I could add the main parser, which
simply parses for a series of class declarations followed by an end-of-file.

