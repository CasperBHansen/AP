%
% 3/solution.tex
%

\subsection{Solution}
I began by defining the API (see appendix \ref{code:sheet|part:api}), notable
design decisions were made for the \code{cell} function. The \code{cell}
function requests a new cell from the spreadsheet, which handles the request
and responds with a cell process ID (this may be an existing cell, and this is
handled by the spreadsheet server, see appendix \ref{code:sheet|part:sheet}).
A sidenote to this is that although \code{set\_value} should be asyncronous
and I call its handler using the syncronous \code{rpc} function, it does
return from the handler immediately after an updater has received its job (see
appendix \ref{code:sheet|part:cell}).

The update server is implemented as a thread that runs in parallel with an
associated cell. An important design decision was made to not keep an updater
alive if the associated cell merely contains a value. This was done by the
case in which the updater does not receive any dependencies (see appendix
\ref{code:sheet|part:update_reset}). If it does indeed have dependencies, the
updater-thread will stay alive, and look for changes in its dependencies, and
accumulate values until a result is acheived. Once a result is computed it
will reset itself. If it is in the process of producing a result and there are
missing data after a period of 500ms it will {\it ping} the cells for a value,
and notify viewers of the associated cell that it is trying to update.

I circumvented the case in which a cell process id doesn't exist yet by simply
creating it, should it be requested by the updater. New cells are given the
value \code{undefined}, which I reason is appropriate in such cases --- that
is, if a cell doesn't exist it must evaluate as undefined for formulae
depending on it.
