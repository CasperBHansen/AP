\section{Generate SVG}
The \texttt{toSVG} function takes a Curve type and generates a string containing
the SVG data.  The \texttt{toFile} function takes a Curve object and and string
with a filename and writes the string returned from \texttt{toSVG} to the file.

In \texttt{toSVG} we first try and convert the Curve into screen coordinates,
where the origin is in the top left of the positive quadrant and not lower left
as in a Kartesian coordinate system.

We generate the header of the SVG using a constant string where the height and
width are calculated by taking the height/width of the curve and then adding
offsets. The lines are created by taking the image height and subtracting the
screen coordinates, to make up for the fact that the y-axis grows downwards in
the screen cooridnate system, and not upwards.

The following imags have been generated from our test Curves using
\texttt{toSVG}.

\graphicc{0.3}{img/test0.png}{Two triangles drawn over one another.}{fig:test0}
\graphicc{0.3}{img/test1.png}{The letter ``M''.}{fig:test1}
\graphicc{0.3}{img/test2.png}{A square, translated slightly left, to test if the
  whitespace would stay.}{fig:test2} \graphicc{0.3}{img/test3.png}{The Figure
  \ref{fig:test0} and Figure \ref{fig:test1} connected using the
  \texttt{Curves->connect} function we implemented.}{fig:test3}
